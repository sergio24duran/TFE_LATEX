\documentclass{article}
\usepackage{graphicx}
\usepackage{fancyhdr}
\usepackage[margin=3.5cm]{geometry} % Margen de 2cm en todos los lados

\pagestyle{fancy}
\fancyhf{}
\fancyhead[L]{\fontsize{12}{14}\selectfont\leftmark}  % Número de página a 
\fancyhead[R]{\fontsize{12}{14}\selectfont\thepage}  % Número de página 
\fancyfoot[R]{\fontsize{12}{14}\selectfont Sevilla, Julio de 2023}  % 


\begin{document}
	
	\begin{titlepage}
		\centering
		\hspace*{-1.5cm}\begin{tabular}{@{}l@{}}
			\includegraphics[width=3cm]{a.png} % Logo 1 (esquina superior izquierda)
		\end{tabular}
		\hfill
		\begin{tabular}{c}
			\LARGE\textbf{Universidad de Sevilla} \\ [0.5cm] % Nombre de la universidad
			\LARGE\textbf{Escuela Politécnica Superior} % Otra frase dentro de la misma caja
		\end{tabular}%
		\hfill
		\begin{tabular}{@{}r@{}}
			\includegraphics[width=3cm]{b.png} % Logo 2 (esquina superior derecha)
		\end{tabular}\hspace*{-1.5cm}
		
		\vspace{1.5cm}
		
		\begin{center}
			\Large\textmd{Trabajo Fin de Grado} \\ [0.5cm] % Título 
			\Large\textmd{Ingeniería Electrónica Industrial}
		\end{center}
		
		\vspace{2cm}
		
		\begin{center}
			\LARGE\textsl{La caracterización integral de las semiaplicaciones de Poincaré y su aplicación a circuitos electrónicos: el memristor}
		\end{center}
		
		\vspace{6cm}
		
		\raggedright
		\large\textbf{Autor:} Sergio R. Durán Martín \\ [0.5cm]
		\large\textbf{Tutor:} Victoriano Carmona Centeno \\ [0.5cm]
		\large\textbf{Departamento:} Matemática Aplicada II
	\end{titlepage}
	
	\newpage
	\section{introduccion}
	hola mundo prueba script
	\newpage
	holaaa
	\newpage
	asldkakdlkad
	\newpage
	\subsection{intro.1}
	holaaa
	\newpage
	\section{alpha}
	segundo cambio tercera prueba
\end{document}
